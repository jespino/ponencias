\documentclass[10pt]{beamer}

\usepackage[utf8]{inputenc}
\usepackage[spanish]{babel}
\usepackage{graphicx}

\mode<presentation>
\usetheme{Madrid}
%\usecolortheme[RGB={111,73,135}]{structure}
\usecolortheme[RGB={128,0,0}]{structure}
%\usecolortheme[RGB={0,96,0}]{structure}
%\usecolortheme[RGB={200,0,200}]{structure}
%\usecolortheme[RGB={0,128,0}]{structure}
%\usecolortheme[RGB={0,0,128}]{structure}
\usefonttheme{serif}
\useinnertheme{rectangles}
\useoutertheme{split}

\setbeamercovered{transparent}

\title{Introducción a Mercurial}
\author{Jesús Espino García}
\date{7 de Noviembre de 2011}
\subject{Introducción a Mercurial}

\institute[Kaleidos]{\includegraphics[height=1.5cm]{kaleidos.png}}

\setcounter{tocdepth}{2}

\AtBeginSubsection[]
{
  \begin{frame}<beamer>{Indice}
    \tableofcontents[sectionstyle=show/shaded,subsectionstyle=show/shaded/hide]
  \end{frame}
}

\begin{document}

  \frame{\maketitle}

  \section*{Introducción}

  \begin{frame}
    \frametitle{¿Que es Mercurial?}
    \begin{itemize}
      \item Sistema de control de versiones.
      \item Distribuido.
      \item Libre.
      \item Python!!.
    \end{itemize}
  \end{frame}

  \begin{frame}
    \frametitle{Ventajas}
  
    \begin{itemize}
      \item Es Open Source.
      \item Es rapido.
      \item Es extensible.
      \item Es potente.
      \item Es (relativamente) sencillo.
      \item Es muy configurable.
    \end{itemize}
  \end{frame}

  \begin{frame}
    \frametitle{Conceptos básicos}
    \begin{itemize}
      \item Repositorio: Contenedor de ficheros y el historico de cambios de los mismos.
      \item Clones: Copias de un determinado repositorio que pueden sincronizarse (o no).
      \item Changeset: Conjunto de cambios relacionados (unidad lógica de almacenamiento de cambios)
      \item Commit: Acción de guardar un changeset.
      \item Pull/Push: Acciones para sincronizar clones (tirar o empujar changesets).
      \item Working directory: Copia de los ficheros del repositorio sobre los que estamos trabajando.
      \item Parent: Cada changeset tiene 0, 1 o 2 padres (inicial, normales y merges).
      \item Rama: Bifurcación del desarrollo (por defecto solo una rama 'default').
    \end{itemize}
  \end{frame}

  \section*{Primera sesión de trabajo}

  \begin{frame}[containsverbatim]
    \frametitle{Obtener el repositorio}
    \begin{itemize}
      \item Localizo la url del repositorio que quiero clonar (en nuestro caso sera ssh://mercurial.kal/curso-mercurial)
      \item Me muevo al directorio donde quiero que se me cree la copia.
      \item Hago un clon del repositorio (para poder trabajar sobre él)
      \item \verb+hg clone ssh://mercurial.kal/curso-mercurial+
      \item Esto me crea un directorio curso-mercurial que tiene un clon del repositorio curso-mercurial.
      \item El directorio contendra los ficheros del último changeset de la rama por defecto y un directorio .hg con todos los datos del repositorio.
    \end{itemize}
  \end{frame}

  \begin{frame}[containsverbatim]
    \frametitle{Configurar mi mercurial}
    \begin{itemize}
      \item Localizo la url del repositorio que quiero clonar (en nuestro caso sera ssh://mercurial.kal/curso-mercurial)
      \item Me muevo al directorio donde quiero que se me cree la copia.
      \item Hago un clon del repositorio (para poder trabajar sobre él)
      \item \verb+hg clone ssh://mercurial.kal/curso-mercurial+
      \item Esto me crea un directorio curso-mercurial que tiene un clon del repositorio curso-mercurial.
      \item El directorio contendra los ficheros del último changeset de la rama por defecto y un directorio .hg con todos los datos del repositorio.
    \end{itemize}
  \end{frame}

  \begin{frame}[containsverbatim]
    \frametitle{Añadir un cambio (commit)}
    \begin{itemize}
      \item Localizo la url del repositorio que quiero clonar (en nuestro caso sera ssh://mercurial.kal/curso-mercurial)
      \item Me muevo al directorio donde quiero que se me cree la copia.
      \item Hago un clon del repositorio (para poder trabajar sobre él)
      \item \verb+hg clone ssh://mercurial.kal/curso-mercurial+
      \item Esto me crea un directorio curso-mercurial que tiene un clon del repositorio curso-mercurial.
      \item El directorio contendra los ficheros del último changeset de la rama por defecto y un directorio .hg con todos los datos del repositorio.
    \end{itemize}
  \end{frame}

  \begin{frame}[containsverbatim]
    \frametitle{Enviar un cambio (push)}
    \begin{itemize}
      \item Localizo la url del repositorio que quiero clonar (en nuestro caso sera ssh://mercurial.kal/curso-mercurial)
      \item Me muevo al directorio donde quiero que se me cree la copia.
      \item Hago un clon del repositorio (para poder trabajar sobre él)
      \item \verb+hg clone ssh://mercurial.kal/curso-mercurial+
      \item Esto me crea un directorio curso-mercurial que tiene un clon del repositorio curso-mercurial.
      \item El directorio contendra los ficheros del último changeset de la rama por defecto y un directorio .hg con todos los datos del repositorio.
    \end{itemize}
  \end{frame}

  \begin{frame}[containsverbatim]
    \frametitle{Obtener cambios (pull)}
    \begin{itemize}
      \item Localizo la url del repositorio que quiero clonar (en nuestro caso sera ssh://mercurial.kal/curso-mercurial)
      \item Me muevo al directorio donde quiero que se me cree la copia.
      \item Hago un clon del repositorio (para poder trabajar sobre él)
      \item \verb+hg clone ssh://mercurial.kal/curso-mercurial+
      \item Esto me crea un directorio curso-mercurial que tiene un clon del repositorio curso-mercurial.
      \item El directorio contendra los ficheros del último changeset de la rama por defecto y un directorio .hg con todos los datos del repositorio.
    \end{itemize}
  \end{frame}

  \begin{frame}[containsverbatim]
    \frametitle{Obtener cambios y mezclar (pull+merge)}
    \begin{itemize}
      \item Localizo la url del repositorio que quiero clonar (en nuestro caso sera ssh://mercurial.kal/curso-mercurial)
      \item Me muevo al directorio donde quiero que se me cree la copia.
      \item Hago un clon del repositorio (para poder trabajar sobre él)
      \item \verb+hg clone ssh://mercurial.kal/curso-mercurial+
      \item Esto me crea un directorio curso-mercurial que tiene un clon del repositorio curso-mercurial.
      \item El directorio contendra los ficheros del último changeset de la rama por defecto y un directorio .hg con todos los datos del repositorio.
    \end{itemize}
  \end{frame}

  \section*{Revisar mi historico (log+diff)}
  \section*{Crear un repositorio nuevo (init)}
  \section*{Mezclando cambios (merge)}
  \section*{Gestionando ficheros (rm+mv+cp+revert)}
    Que guardar y que no
  \section*{Ramas (branch)}
  \section*{Mezclando ramas (merge)}
  \section*{Ignorando ficheros (.hgingore)}
  \section*{Deshaciendo cosas}
  \section*{Hooks (que son)}
  \section*{Bookmarks}
  \section*{Subrepositorios}
  \section*{Extensiones (instalar y configurar)}

  \section*{Para terminar}

  \begin{frame}
    \frametitle{Referencias}
    \begin{itemize}
      \item \small{http://mercurial.selenic.com/ - Web oficial de mercurial.}
      \item \small{http://hgbook.red-bean.com/ - Mercurial: The definitive guide.}
      \item \small{https://bitbucket.org/ - Bit Bucket}
    \end{itemize}
  \end{frame}

  \begin{frame}
    \frametitle{Dudas}
    \dots
  \end{frame}

\end{document}
