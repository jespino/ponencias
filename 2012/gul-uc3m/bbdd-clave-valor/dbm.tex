\documentclass[10pt]{beamer}

\usepackage[utf8]{inputenc}
\usepackage[spanish]{babel}
\usepackage{graphicx}

\mode<presentation>
\usetheme{Madrid}
%\usecolortheme[RGB={111,73,135}]{structure}
\usecolortheme[RGB={128,0,0}]{structure}
%\usecolortheme[RGB={0,96,0}]{structure}
%\usecolortheme[RGB={200,0,200}]{structure}
%\usecolortheme[RGB={0,128,0}]{structure}
%\usecolortheme[RGB={0,0,128}]{structure}
\usefonttheme{serif}
\useinnertheme{rectangles}
\useoutertheme{split}

\setbeamercovered{transparent}

\title{Bases de datos Clave/Valor}
\author{Jesús Espino García}
\date{13 de Marzo de 2012}
\subject{Bases de datos Clave/Valor}

\institute[Kaleidos]{\includegraphics[height=1.5cm]{kaleidos.png}}

\setcounter{tocdepth}{2}

\AtBeginSubsection[]
{
  \begin{frame}<beamer>{Indice}
    \tableofcontents[sectionstyle=show/shaded,subsectionstyle=show/shaded/hide]
  \end{frame}
}

\begin{document}

  \frame{\maketitle}

  \section*{Introducción}

  \begin{frame}
    \frametitle{¿Que son?}
    \begin{itemize}
      \item Bases de datos.
      \item NoSQL (Pre-Relacional).
      \item Basadas en dbm (Ken Thompson, 1979).
      \item Almacenan pares de Clave/Valor.
      \item Muy rápidas.
      \item Relativamente poca funcionalidad.
    \end{itemize}
  \end{frame}

  \begin{frame}
    \frametitle{¿Dónde se usan?}
    En todas partes, algunos ejemplos:
    \begin{itemize}
      \item Postfix
      \item MySQL
      \item OpenLDAP
      \item SpamAssassin
      \item Subversion
      \item RPM
      \item Asterisk
      \item KDevelop
      \item ...
    \end{itemize}
  \end{frame}

  \begin{frame}
    \frametitle{Ventajas}
  
    \begin{itemize}
      \item Son más sencillas de entender.
      \item Se trabaja más a bajo nivel, pudiendo optimizar más.
      \item Se elimina el overhead de los RDBMS.
      \item Fácil de usar en modelos de datos sencillos.
    \end{itemize}
  \end{frame}

  \begin{frame}
    \frametitle{Inconvenientes}
  
    \begin{itemize}
      \item Menor funcionalidad.
      \item No hay un sistema de consultas genérico.
      \item Complicado de usar con modelos de datos complejos.
    \end{itemize}
  \end{frame}

  \begin{frame}[containsverbatim]
    \frametitle{Algunas implementaciones}
    \begin{itemize}
      \item DBM: Implementación original.
      \item BerkeleyDB: La más extendida, mantenida por Oracle.
      \item JDBM/JDBM2: Implementación Java.
      \item GDBM: Implementación de GNU.
      \item QDBM/TokyoCabinet/KyotoCabinet: Implementaciones de Mikio Hirabayashi.
    \end{itemize}
  \end{frame}

  \section*{Trabajando con bases de datos}

  \begin{frame}[containsverbatim]
    \frametitle{Escoger el tipo de base de datos}
    \begin{itemize}
      \item Importante elegir bien.
      \item Depende de las tipos que de mi DBM.
      \item Normalmente B+Tree o Hash.
      \item Depende de muchos factores:
      \begin{itemize}
        \item balance lecturas/escrituras.
        \item tipo de consultas (clave, rango).
        \item tamaño del conjunto de datos.
        \item implementación concreta del DBM.
      \end{itemize}
    \end{itemize}
  \end{frame}

  \begin{frame}
    \frametitle{Ejemplos}
    \begin{itemize}
      \item Ejemplos con KyotoCabinet
    \end{itemize}
  \end{frame}

  \section*{Para terminar}

  \begin{frame}
    \frametitle{Referencias}
    \begin{itemize}
      \item \small{http://http://fallabs.com/ - FAL Labs (QDBM, TokyoCabinet and KyotoCabinet creators)}
      \item \small{http://www.oracle.com/technetwork/products/berkeleydb/overview/index.html - BerkeleyDB}
    \end{itemize}
  \end{frame}

  \begin{frame}
    \frametitle{Dudas}
    \dots
  \end{frame}

\end{document}
