\documentclass[10pt]{beamer}

\usepackage[utf8]{inputenc}
\usepackage[spanish]{babel}
\usepackage{graphicx}

\mode<presentation>
\usetheme{Madrid}
%\usecolortheme[RGB={111,73,135}]{structure}
\usecolortheme[RGB={128,0,0}]{structure}
%\usecolortheme[RGB={0,96,0}]{structure}
%\usecolortheme[RGB={200,0,200}]{structure}
%\usecolortheme[RGB={0,128,0}]{structure}
%\usecolortheme[RGB={0,0,128}]{structure}
\usefonttheme{serif}
\useinnertheme{rectangles}
\useoutertheme{split}

\setbeamercovered{transparent}

\title{Introducción a Mercurial}
\author{Jesús Espino García}
\date{28 de Enero de 2012}
\subject{Introducción a Mercurial}

\institute[Kaleidos]{\includegraphics[height=1.5cm]{kaleidos.png}}

\setcounter{tocdepth}{2}

\AtBeginSubsection[]
{
  \begin{frame}<beamer>{Indice}
    \tableofcontents[sectionstyle=show/shaded,subsectionstyle=show/shaded/hide]
  \end{frame}
}

\begin{document}

  \frame{\maketitle}

  \section*{Introducción}

  \begin{frame}
    \frametitle{¿Que es Mercurial?}
    \begin{itemize}
      \item Sistema de control de versiones.
      \item Distribuido.
      \item Libre.
      \item Python.
    \end{itemize}
  \end{frame}

  \begin{frame}
    \frametitle{Ventajas}
  
    \begin{itemize}
      \item Es Open Source.
      \item Es rápido.
      \item Es extensible.
      \item Es potente.
      \item Es (relativamente) sencillo.
      \item Es muy configurable.
    \end{itemize}
  \end{frame}

  \begin{frame}
    \frametitle{Conceptos básicos}
    \begin{itemize}
      \item Repositorio: Contenedor de ficheros y el histórico de cambios de los mismos.
      \item Clones: Copias de un determinado repositorio que pueden sincronizarse (o no).
      \item Changeset: Conjunto de cambios relacionados (unidad lógica de almacenamiento de cambios)
      \item Commit: Acción de guardar un changeset.
      \item Pull/Push: Acciones para sincronizar clones (tirar o empujar changesets).
      \item Working directory: Copia de los ficheros del repositorio sobre los que estamos trabajando.
      \item Rama: Bifurcación del desarrollo (por defecto solo una rama 'default').
    \end{itemize}
  \end{frame}

  \section*{Primera sesión de trabajo}

  \begin{frame}[containsverbatim]
    \frametitle{Crear un repositorio}
    \begin{itemize}
      \item Crear un repositorio (hg init nombre-del-repositorio).
      \item Entrar en el directorio del repositorio (cd nombre-del-repositorio)
    \end{itemize}
  \end{frame}

  \begin{frame}[containsverbatim]
    \frametitle{Añadir mis primeros ficheros}
    \begin{itemize}
      \item Copiar o crear los ficheros que queramos incluir.
      \item Añadir los ficheros al control de versiones (hg add *).
      \item Guardamos los cambios hechos hasta ahora (hg commit).
    \end{itemize}
  \end{frame}

  \begin{frame}[containsverbatim]
    \frametitle{Añadir un cambio (commit)}
    \begin{itemize}
      \item Modificamos los ficheros.
      \item Guardamos los cambios hechos hasta ahora (hg commit).
    \end{itemize}
  \end{frame}

  \section*{Revisar mi histórico}

  \begin{frame}[containsverbatim]
    \frametitle{Viendo mi histórico}
    \begin{itemize}
      \item Ver la historia de mi proyecto (hg log).
    \end{itemize}
  \end{frame}

  \begin{frame}[containsverbatim]
    \frametitle{Comparando con el pasado}
    \begin{itemize}
        \item Que he cambiado desde la ultima vez que guarde (hg diff)
        \item Que he cambiado desde un determinado momento (hg diff -r XXXXX)
    \end{itemize}
  \end{frame}

  \begin{frame}[containsverbatim]
    \frametitle{Regreso al pasado}
    \begin{itemize}
        \item Volviendo a un momento del proyecto (hg checkout -r XXXXX)
    \end{itemize}
  \end{frame}

  \section*{Mercurial distribuido}

  \begin{frame}[containsverbatim]
    \frametitle{Clonar un repositorio}
    \begin{itemize}
      \item Buscamos la url del repositorio a clonar.
      \item Clonamos el repositorio (hg clone URL).
    \end{itemize}
  \end{frame}

  \begin{frame}[containsverbatim]
    \frametitle{Obtener cambios de un repositorio remoto}
    \begin{itemize}
      \item Obtener cambios de un repositorio remoto (hg pull).
      \item Si tengo cambios míos, mezclarlos con los obtenidos (hg merge + hg commit).
      \item Si no tengo cambios míos, actualizo mi copia de trabajo (hg update).
    \end{itemize}
  \end{frame}

  \begin{frame}[containsverbatim]
    \frametitle{Enviar cambios a un repositorio remoto}
    \begin{itemize}
      \item Comprobamos si hay cambios en repositorio remoto (hg pull).
      \item Si hay cambios, los mezclamos (hg merge + hg commit).
      \item Enviamos los cambios al repositorio remoto (hg push).
    \end{itemize}
  \end{frame}

  \begin{frame}[containsverbatim]
    \frametitle{Conflictos}
    \begin{itemize}
      \item Al mezclar pueden darse conflictos (cambios incompatibles).
      \item Un cambio será incompatible con otro, si modifican el mismo fichero en las mismas lineas.
      \item El último en llegar se encarga de arreglar el conflicto.
      \item Mercurial nos da ciertas ayudas para arreglarlos.
      \item Una vez arreglados los conflictos hay que marcar los ficheros como resueltos (hg resolve -m fichero).
      \item Tras esto se guardan los cambios (hg commit).
    \end{itemize}
  \end{frame}

  \section*{Ramas}

  \begin{frame}[containsverbatim]
    \frametitle{Creando una rama}
    \begin{itemize}
      \item Creamos una rama (hg branch nuevarama)
      \item Mercurial automáticamente nos cambia a esta rama.
      \item Los commits se harán en esta rama a partir de ahora.
      \item Podemos listar las ramas con hg branches.
    \end{itemize}
  \end{frame}

  \begin{frame}[containsverbatim]
    \frametitle{Cambiando de rama}
    \begin{itemize}
      \item Cambiamos de rama (hg update default)
      \item Los commits se harán en esta rama a partir de ahora.
    \end{itemize}
  \end{frame}

  \begin{frame}[containsverbatim]
    \frametitle{Mezclando ramas}
    \begin{itemize}
      \item Me sitúo en la rama donde quiero mezclar los cambios (hg update default).
      \item Mezclo los cambios de otra rama (hg merge nuevarama).
      \item Tras esto guardo los cambios (hg commit).
    \end{itemize}
  \end{frame}

  \section*{Ignorando ficheros}

  \begin{frame}[containsverbatim]
    \frametitle{Ignorando ficheros}
    \begin{itemize}
      \item Creamos el fichero .hgignore
      \item Incluimos lineas que definan que ficheros ignorar (*.txt)
    \end{itemize}
  \end{frame}

  \section*{Para terminar}

  \begin{frame}
    \frametitle{Que se nos queda en el tintero}
    \begin{itemize}
      \item Hooks
      \item Bookmarks
      \item Subrepositorios
      \item Queues
      \item Extensiones
    \end{itemize}
  \end{frame}

  \begin{frame}
    \frametitle{Referencias}
    \begin{itemize}
      \item \small{http://mercurial.selenic.com/ - Web oficial de mercurial.}
      \item \small{http://hgbook.red-bean.com/ - Mercurial: The definitive guide.}
      \item \small{https://bitbucket.org/ - Bit Bucket}
    \end{itemize}
  \end{frame}

  \begin{frame}
    \frametitle{Dudas}
    \dots
  \end{frame}

\end{document}
