\documentclass[10pt]{beamer}

\usepackage[latin1]{inputenc}
\usepackage[spanish]{babel}

\mode<presentation>
\usetheme{Madrid}
\usecolortheme[RGB={128,0,0}]{structure}
\usefonttheme{serif}
\useinnertheme{rectangles}
\useoutertheme{split}
\setbeamercovered{transparent}

\title{AngularJS Internals}
\author{Jes�s Espino Garc�a}
\date{30 de Abril de 2014}
\subject{AngularJS Internals}

\institute[Kaleidos]{\includegraphics[height=1.5cm]{kaleidos}}

\begin{document}

  \frame{\maketitle}

  \begin{frame}{�Por qu�?}
    \begin{itemize}
      \item La magia que no entendemos es magia negra.
      \item Angular es maravillos hasta que te muerde.
      \item Saber como funciona permite sacarle el m�ximo partido.
    \end{itemize}
  \end{frame}

  \begin{frame}{�C�mo vamos a verlo?}
    \begin{itemize}
      \item Describiremos los pasos que da Angular para hacer algo.
      \item Mostraremos el codigo que lo hace.
      \item Pasaremos con el debugger sobre una aplicacion de ejemplo.
      \item Para todo ello usaremos el git de angular en su tag v1.2.16.
    \end{itemize}
  \end{frame}

  \begin{frame}{�Como arranca angularJS?}
    \begin{itemize}
      \item Se ejecuta angularInit (angular.suffix:14)
      \item angularInit busca opciones para inicialilizar una app (Angular.js:1188).
      \item Ejecuta bootstrap sobre el elemento (Angular.js:1228).
      \item Se a�aden los modulos ng y \$provide al listado de modulos (Angular.js:1296).
      \item Se crea el injector con el listado de modulos (Angular.js:1300).
      \item Y se invoca la compilaci�n y el "digest" del contenido del elemento app (Angular.js:1303).
    \end{itemize}
  \end{frame}

  \begin{frame}{�Que es el injector?}
    \begin{itemize}
      \item Permite hacer injecci�n de dependencias.
      \item Parsea la funcion para extrar el nombre de los parametros que recibe (auto/injector.js:69).
      \item Sustituye los parametros que recibe por instancias de servicios (auto/injector.js:746).
      \item Al crearse carga los modulos necesesarios (auto/injector.js:675). ### REVISAR
      \item Tras cargar los modulos, los instancia (auto/injector.js:617).
    \end{itemize}
  \end{frame}

  \begin{frame}{�C�mo funciona la compilaci�n?}
    \begin{itemize}
      \item 
      \item 
      \item 
      \item 
      \item 
    \end{itemize}
  \end{frame}

  \begin{frame}{�Como funciona el data-binding?}
    \begin{itemize}
      \item 
      \item 
      \item 
      \item 
      \item 
    \end{itemize}
  \end{frame}

  \begin{frame}{�Que es un servicio?}
    \begin{itemize}
      \item 
      \item 
      \item 
      \item 
      \item 
    \end{itemize}
  \end{frame}

  \begin{frame}{�Que es un filtro?}
    \begin{itemize}
      \item 
      \item 
      \item 
      \item 
      \item 
    \end{itemize}
  \end{frame}

  \begin{frame}{�Que es una directiva?}
    \begin{itemize}
      \item 
      \item 
      \item 
      \item 
      \item 
    \end{itemize}
  \end{frame}

  \begin{frame}{Dudas}
    \dots
  \end{frame}

  \begin{frame}
    \frametitle{\begin{center}Fin\end{center}}
    \begin{center}
      \includegraphics[height=3cm]{kaleidos}
    \end{center}
  \end{frame}

\end{document}
