\documentclass[10pt]{beamer}

\usepackage[latin1]{inputenc}
\usepackage[spanish]{babel}

\mode<presentation>
\usetheme{Madrid}
%\usecolortheme[RGB={111,73,135}]{structure}
\usecolortheme[RGB={128,0,0}]{structure}
%\usecolortheme[RGB={0,96,0}]{structure}
%\usecolortheme[RGB={200,0,200}]{structure}
%\usecolortheme[RGB={0,128,0}]{structure}
%\usecolortheme[RGB={0,0,128}]{structure}
\usefonttheme{serif}
\useinnertheme{rectangles}
\useoutertheme{split}
\setbeamercovered{transparent}


% Definiciones para usar luego
\newtheorem{ejemplo}{Ejemplo}
\newtheorem{definicion}{Definici�n}

\title{Git intermedio: M�s all� del commit}
\author{Jes�s Espino Garc�a}
\date{4 de Abril de 2014}
\subject{Git intermedio}

\institute[Kaleidos]{\includegraphics[height=1.5cm]{kaleidos}}

\begin{document}

  \frame{\maketitle}
  
  \section{Introducci�n}

  \begin{frame}{�Por qu�?}
    \begin{itemize}
      \item Es una herramienta importante.
      \item La usamos a diario.
      \item Es tambien una base de conocimiento.
    \end{itemize}
  \end{frame}
  
  \begin{frame}{�Qu� supongo que ya sabeis?}
    \begin{itemize}
      \item Add/Commit
      \item Merge
      \item Fetch/Pull/Push
      \item Ramas
      \item Logs
    \end{itemize}
  \end{frame}

  \begin{frame}{�Qu� vamos a ver?}
    \begin{itemize}
      \item Index
      \item Reflog
      \item Commit/Add patch
      \item Commit/Range selectors
      \item Bisect
      \item Cherry-picking
      \item Rebases
      \item Submodules
    \end{itemize}
  \end{frame}

  \section{Index}
  \section{Reflog}
  \section{Commit/Add patch}
  \section{Commit/Range selectors}
  \section{Bisect}
  \section{Cherry-picking}
  \section{Rebases}
  \section{Submodules}

  \section{Para terminar.}

  \begin{frame}{Dudas}
    \dots
  \end{frame}
  
  \begin{frame}
    \frametitle{\begin{center}Fin\end{center}}
    \begin{center}
      \includegraphics[height=3cm]{kaleidos}
    \end{center}
  \end{frame}

\end{document}
