\documentclass[10pt]{beamer}

\usepackage[latin1]{inputenc}
\usepackage[spanish]{babel}

\mode<presentation>
\usetheme{Madrid}
\usecolortheme[RGB={128,0,0}]{structure}
\usefonttheme{serif}
\useinnertheme{rectangles}
\useoutertheme{split}
\setbeamercovered{transparent}

\title{Git: M�s all� del commit}
\author{Jes�s Espino Garc�a}
\date{30 de Abril de 2014}
\subject{Git: M�s all� del commit}

\institute[Kaleidos]{\includegraphics[height=1.5cm]{kaleidos}}

\begin{document}

  \frame{\maketitle}

  \begin{frame}{�Por qu�?}
    \begin{itemize}
      \item Es una herramienta importante.
      \item La usamos a diario.
      \item Es tambien una base de conocimiento.
    \end{itemize}
  \end{frame}

  \begin{frame}{�Qu� supongo que ya sabeis?}
    \begin{itemize}
      \item Add/Commit
      \item Merge
      \item Fetch/Pull/Push
      \item Ramas
      \item Tags
      \item Logs
    \end{itemize}
  \end{frame}

  \begin{frame}{�Qu� vamos a ver?}
    \begin{itemize}
      \item Index
      \item Reflog
      \item Commit/Add patch
      \item Commit/Range selectors
      \item Checkout
      \item Reset
      \item Bisect
      \item Revert
      \item Cherry-picking
      \item Rebases
      \item Aliases
      \item Stash
      \item Archive
      \item Bundle
      \item Gitignore
      \item Hooks
    \end{itemize}
  \end{frame}

  \begin{frame}{Index}
    \begin{itemize}
      \item El lugar para poner lo que vas a comitear
      \item Al a�adir un cambio o fichero al index se genera un blob nuevo.
      \item Puede haber cambios diferentes en la copia de trabajo y el index.
    \end{itemize}
  \end{frame}

  \begin{frame}[containsverbatim]{Index}
    \begin{verbatim}
      git status
      git add file-with-changes.txt
      git status
      echo "example" >> file-with-changes.txt
      git status
    \end{verbatim}
  \end{frame}

  \begin{frame}{Reflog}
    \begin{itemize}
      \item Log de referencias.
      \item Se a�ade una entrada cada vez que cambia HEAD.
      \item Es util para acceder a commits desreferenciados.
      \item Suele ser util para ver en que ramas has hecho ciertas cosas.
    \end{itemize}
  \end{frame}

  \begin{frame}[containsverbatim]{Reflog}
    \begin{verbatim}
      git reflog
      git checkout -b new-branch
      git checkout master
      git commit --allow-empty -m "example"
      git checkout new-branch
      git checkout master
      git reflog
    \end{verbatim}
  \end{frame}

  \begin{frame}{Commit/Add patch}
    \begin{itemize}
      \item Es importante la atomicidad de los commits.
      \item Se puede hacer a�adido o commits de partes de un fichero.
    \end{itemize}
  \end{frame}

  \begin{frame}[containsverbatim]{Commit/Add patch}
    \begin{verbatim}
      git add -p
      git commit -p
    \end{verbatim}
  \end{frame}

  \begin{frame}{Commit/Range selectors}
    \begin{itemize}
      \item Hay que ser capaz de referenciar commits o rangos.
      \item Acceder a commits por diferentes caminos.
    \end{itemize}
  \end{frame}

  \begin{frame}[containsverbatim]{Commit/Range selectors}
    \begin{itemize}
      \item Acceder a un commit concreto
      \begin{itemize}
        \item \verb+HEAD+
        \item ramas (ejemplo: \verb+master+)
        \item tags (ejemplo: \verb+v1.0.0+)
        \item commit id (ejemplo: \verb+f7234b124a+)
      \end{itemize}
    \end{itemize}
  \end{frame}

  \begin{frame}[containsverbatim]{Commit/Range selectors}
    \begin{itemize}
      \item Acceder a commit de manera relativa
      \begin{itemize}
        \item \verb+<refname>@{<date>}+ (ejemplos: \verb+<refname>@{yesterday}+, \verb+<refname>@{5 minutes ago}+, \verb+master@{1979-02-26}+)
        \item \verb+[<refname>]@{<n>}+ (ejemplos: \verb+master@{2}, @{2}+)
        \item \verb+[<refname>]@{upstream}+ (ejemplo: \verb+master@{upstream}+)
        \item \verb+<rev>\^, <rev>\^<n>+ (ejemplo: \verb+HEAD\^2+)
        \item \verb+<rev>~<n>+ (ejemplo: \verb+master~3+)
        \item \verb+<rev>\^{/<text>}+, :/text (ejemplo: \verb+<rev>\^{/model}+)
        \item \verb+<rev>:<path>+ (ejemplo: \verb+master:settings.py+)
      \end{itemize}
      \item ver \verb+man gitrevisions+
    \end{itemize}
  \end{frame}

  \begin{frame}[containsverbatim]{Commit/Range selectors}
    \begin{verbatim}
      git show master@{yesterday}
      git show master@{3}
      git show master@{upstream}
      git show master^
      git show master^2
      git show master~3
      git show master~3^2~5
      git show master^{/text}
      git show :/text
      git show master:settings.py
    \end{verbatim}
  \end{frame}

  \begin{frame}[containsverbatim]{Commit/Range selectors}
    \begin{itemize}
      \item Acceder a rangos
      \begin{itemize}
        \item \verb+<rev>+ (ejemplo: \verb+git log HEAD~5+)
        \item \verb+<rev1>..<rev2>+ (ejemplo: \verb+git log master~5..master+)
        \item \verb+<rev1>..<rev2>+ (ejemplo: \verb+git log master...stable+)
        \item \verb+<rev>\^@+ (ejemplo: \verb+git log HEAD\^@+)
        \item \verb+<rev>\^!+ (ejemplo: \verb+git log HEAD\^!+)
      \end{itemize}
      \item ver \verb+man gitrevisions+
    \end{itemize}
  \end{frame}

  \begin{frame}[containsverbatim]{Commit/Range selectors}
    \begin{verbatim}
      git log master~3
      git log master~3..master
      git log master...stable
      git log master^@
      git log master^!
    \end{verbatim}
  \end{frame}

  \begin{frame}{Checkout}
    \begin{itemize}
      \item Saca una version del fichero a la working copy
      \item Util para descartar cambios de un fichero concreto.
      \item Util para restaurar el estado de un fichero en otra revision.
    \end{itemize}
  \end{frame}

  \begin{frame}[containsverbatim]{Checkout}
    \begin{verbatim}
      # discard changes on settings.py
      git checkout settings.py

      # Get the previous version of settings.py
      git checkout HEAD~1 settings.py

      # Get chunks of the previous version of settings.py
      git checkout -p HEAD~1 settings.py
    \end{verbatim}
  \end{frame}

  \begin{frame}{Reset}
    \begin{itemize}
      \item Permite restablecer a un commit la rama, el index y la copia de trabajo.
      \item Existen 5 modos de reset:
      \begin{itemize}
        \item soft: Establece solo el puntero de la rama
        \item mixed: Establece el puntero y reinicia el index (Por defecto).
        \item hard: Establece el puntero y reinicia el index y la copia de trabajo.
        \item merge: Como un hard pero conserva los cambios que no est�n en el index.
        \item keep: Como hard pero mateniendo los ficheros cambiados.
      \end{itemize}
      \item Los modos habituales son soft, mixed y hard.
    \end{itemize}
  \end{frame}

  \begin{frame}[containsverbatim]{Reset}
    \begin{verbatim}
      # Uncommit
      git reset --soft HEAD^

      # Cleanup the working copy and index
      git reset --hard

      # Imitaci�n de git reset --keep HEAD~5
      git stash; git reset --hard HEAD~5; git stash pop
    \end{verbatim}
  \end{frame}

  \begin{frame}{Bisect}
    \begin{itemize}
      \item Busqueda dicotomica de errores.
      \item Permite detectar exactamente en que commit empezo a ocurrir algo.
      \item Se puede hacer manualmente o de manera automatica.
    \end{itemize}
  \end{frame}

  \begin{frame}[containsverbatim]{Bisect}
    Manualmente
    \begin{verbatim}
      git bisect start
      git bisect bad
      git bisect good master~10
      # Checking the problem (fail?)
      git bisect bad
      # Checking the problem (works?)
      git bisect good
      # Checking the problem (works?)
      git bisect good
    \end{verbatim}
  \end{frame}

  \begin{frame}[containsverbatim]{Bisect}
    Automaticamente
    \begin{verbatim}
      git bisect start
      git bisect bad
      git bisect good master~10
      git bisect run test.sh
    \end{verbatim}
  \end{frame}

  \begin{frame}{Revert}
    \begin{itemize}
      \item Es interesante poder deshacer un commit.
      \item Es la mejor opci�n cuando tus cambios ya subidos.
    \end{itemize}
  \end{frame}

  \begin{frame}[containsverbatim]{Revert}
    \begin{verbatim}
      git revert :/Bug#37
    \end{verbatim}
  \end{frame}

  \begin{frame}{Cherry-picking}
    \begin{itemize}
      \item Es interesante poder reaplicar un commit en otra rama.
      \item No siempre quieres hacer un merge.
    \end{itemize}
  \end{frame}

  \begin{frame}[containsverbatim]{Cherry-picking}
    \begin{verbatim}
      git cherry-pick 1b35539
      git cherry-pick stable~3..stable
    \end{verbatim}
  \end{frame}

  \begin{frame}{Rebases}
    \begin{itemize}
      \item Permiten cambiar mis commits.
      \item Reorganizarlos.
      \item Actualizar mi rama de una manera mas limpia.
      \item Mover un cojunto de commits a otro punto.
    \end{itemize}
  \end{frame}

  \begin{frame}{Rebases de ramas}
    \begin{itemize}
      \item Se utiliza para actualizar la rama actual.
      \item Funciona en tres pasos:
      \begin{itemize}
        \item Elimina los commits aplicados en la rama.
        \item Coloca el puntero de la rama sobre el ultimo commit de la otra rama.
        \item A�ade los commits eliminados previamente.
      \end{itemize}
    \end{itemize}
  \end{frame}

  \begin{frame}[containsverbatim]{Rebases de ramas}
    \begin{verbatim}
      git checkout my-branch
      # Add some commits
      git fetch
      git rebase origin/master
    \end{verbatim}
  \end{frame}

  \begin{frame}{Rebases sobre la misma rama}
    \begin{itemize}
      \item Es interesante poder reestructurar mis commits.
      \item Reordenarlos.
      \item Juntarlos.
      \item Renomrar el mensaje de commit.
    \end{itemize}
  \end{frame}

  \begin{frame}[containsverbatim]{Rebases sobre la misma rama}
    \begin{verbatim}
      git rebase -i master
    \end{verbatim}
  \end{frame}

  \begin{frame}{Aliases}
    \begin{itemize}
      \item Pueden crearse alias de comandos.
      \item Nos ahorra algo de tiempo.
      \item Nos facilita no tener que recordar ciertas cosas.
    \end{itemize}
  \end{frame}

  \begin{frame}[containsverbatim]{Aliases}
    \begin{verbatim}
      git alias uncommit "git reset --soft HEAD^"
      git alias
      git alias unco
    \end{verbatim}
  \end{frame}

  \begin{frame}{Stash}
    \begin{itemize}
      \item Almacena cambios no comiteados en una pila.
      \item Muy util para dejar trabajo a medias y retomarlo luego.
    \end{itemize}
  \end{frame}

  \begin{frame}[containsverbatim]{Stash}
    \begin{verbatim}
      git stash
      git stash list
      git stash pop

      # Better save
      git stash save Trabajo parcial en funcionalidad X
      git stash save -p Trabajo parcial en funcionalidad X
    \end{verbatim}
  \end{frame}

  \begin{frame}{Archive}
    \begin{itemize}
      \item Permite crear un tar/zip del repo.
      \item Util para generar versiones.
      \item Util para despliegues.
    \end{itemize}
  \end{frame}

  \begin{frame}[containsverbatim]{Archive}
    \begin{verbatim}
      git archive master --format=tar --output=master.tar
    \end{verbatim}
  \end{frame}

  \begin{frame}{Bundle}
    \begin{itemize}
      \item Permite crear un archivo que actua como repo.
      \item Util para compartir un repo.
      \item Se puede compartir solo ramas concretas.
    \end{itemize}
  \end{frame}

  \begin{frame}[containsverbatim]{Bundle}
    \begin{verbatim}
      git bundle mi-repo.git master stable
    \end{verbatim}
  \end{frame}

  \begin{frame}{Gitignore}
    \begin{itemize}
      \item Puedes ignorar ficheros.
      \item por ejemplo los .pyc o .class.
      \item Patrones en .gitignore o en .git/info/exclude
    \end{itemize}
  \end{frame}

  \begin{frame}[containsverbatim]{Gitignore}
    \begin{verbatim}
      *.pyc
      *.class
      !*.py
      directory/
      all/directories/**/
      all/files/**
    \end{verbatim}
  \end{frame}

  \begin{frame}{Hooks}
    \begin{itemize}
      \item Acciones ante ciertos eventos
      \item Normalmente Pre/Post Commit/Push
      \item Ejemplos habituales
      \begin{itemize}
          \item Analisis de covertura/estilos (cliente o servidor).
          \item Despliegue.
          \item Integraci�n continua.
      \end{itemize}
    \end{itemize}
  \end{frame}

  \begin{frame}{Dudas}
    \dots
  \end{frame}

  \begin{frame}
    \frametitle{\begin{center}Fin\end{center}}
    \begin{center}
      \includegraphics[height=3cm]{kaleidos}
    \end{center}
  \end{frame}

\end{document}
