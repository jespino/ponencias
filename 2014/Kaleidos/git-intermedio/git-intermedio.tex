\documentclass[10pt]{beamer}

\usepackage[latin1]{inputenc}
\usepackage[spanish]{babel}

\mode<presentation>
\usetheme{Madrid}
%\usecolortheme[RGB={111,73,135}]{structure}
\usecolortheme[RGB={128,0,0}]{structure}
%\usecolortheme[RGB={0,96,0}]{structure}
%\usecolortheme[RGB={200,0,200}]{structure}
%\usecolortheme[RGB={0,128,0}]{structure}
%\usecolortheme[RGB={0,0,128}]{structure}
\usefonttheme{serif}
\useinnertheme{rectangles}
\useoutertheme{split}
\setbeamercovered{transparent}


% Definiciones para usar luego
\newtheorem{ejemplo}{Ejemplo}
\newtheorem{definicion}{Definici�n}

\title{Git intermedio: M�s all� del commit}
\author{Jes�s Espino Garc�a}
\date{4 de Abril de 2014}
\subject{Git intermedio}

\institute[Kaleidos]{\includegraphics[height=1.5cm]{kaleidos}}

\begin{document}

  \frame{\maketitle}

  \section{Introducci�n}

  \begin{frame}{�Por qu�?}
    \begin{itemize}
      \item Es una herramienta importante.
      \item La usamos a diario.
      \item Es tambien una base de conocimiento.
    \end{itemize}
  \end{frame}

  \begin{frame}{�Qu� supongo que ya sabeis?}
    \begin{itemize}
      \item Add/Commit
      \item Merge
      \item Fetch/Pull/Push
      \item Ramas
      \item Logs
    \end{itemize}
  \end{frame}

  \begin{frame}{�Qu� vamos a ver?}
    \begin{itemize}
      \item Index
      \item Reflog
      \item Commit/Add patch
      \item Commit/Range selectors
      \item Bisect
      \item Revert
      \item Cherry-picking
      \item Rebases
      \item Reset
    \end{itemize}
  \end{frame}

  \section{Index}

  \begin{frame}{Index}
    \begin{itemize}
      \item El lugar para poner lo que vas a comitear
      \item Al a�adir un cambio o fichero al index se genera un blob nuevo.
      \item Puede haber cambios diferentes en la copia de trabajo y el index.
    \end{itemize}
  \end{frame}

  \section{Reflog}

  \begin{frame}{Reflog}
    \begin{itemize}
      \item Log de referencias.
      \item Se a�ade una entrada cada vez que cambia HEAD.
      \item Es util para acceder a commits desreferenciados.
      \item Suele ser util para ver en que ramas has hecho ciertas cosas.
    \end{itemize}
  \end{frame}

  \section{Commit/Add patch}

  \begin{frame}{Commit/Add patch}
    \begin{itemize}
      \item Es importante la atomicidad de los commits.
      \item Se puede hacer a�adido o commits de partes de un fichero.
    \end{itemize}
  \end{frame}

  \section{Commit/Range selectors}

  \begin{frame}{Commit/Range selectors}
    \begin{itemize}
      \item Hay que ser capaz de referenciar commits o rangos.
      \item Acceder a commits por diferentes caminos.
    \end{itemize}
  \end{frame}

  \section{Bisect}

  \begin{frame}{Bisect}
    \begin{itemize}
      \item Busqueda dicotomica de errores.
      \item Permite detectar exactamente en que commit empezo a ocurrir algo.
      \item Se puede hacer manualmente o de manera automatica.
    \end{itemize}
  \end{frame}

  \section{Revert}

  \begin{frame}{Revert}
    \begin{itemize}
      \item Es interesante poder deshacer un commit.
    \end{itemize}
  \end{frame}

  \section{Cherry-picking}

  \begin{frame}{Cherry-picking}
    \begin{itemize}
      \item Es interesante poder reaplicar un commit en otra rama.
      \item No siempre quieres hacer un merge.
    \end{itemize}
  \end{frame}

  \section{Rebases}

  \begin{frame}{Rebases de ramas}
    \begin{itemize}
      \item Se utiliza para actualizar la rama actual.
      \item Funciona en tres pasos:
      \begin{itemize}
        \item Elimina los commits aplicados en la rama.
        \item Coloca el puntero de la rama sobre el ultimo commit de la otra rama.
        \item A�ade los commits eliminados previamente.
      \end{itemize}
    \end{itemize}
  \end{frame}

  \begin{frame}{Rebases sobre la misma rama}
    \begin{itemize}
      \item Es interesante poder reestructurar mis commits.
      \item Reordenarlos.
      \item Juntarlos.
      \item Renomrar el mensaje de commit.
    \end{itemize}
  \end{frame}

  \section{Reset}

  \begin{frame}{Reset}
    \begin{itemize}
      \item Permite modificar el puntero de la rama.
      \item Existen 5 modos de reset:
      \begin{itemize}
        \item soft: Establece solo el puntero de la rama
        \item mixed: Establece el puntero y reinicia el index.
        \item hard: Establece el puntero y reinicia el index y la copia de trabajo.
        \item merge: Como un hard pero conserva los cambios que est�n en el index.
        \item keep: Como hard pero mateniendo los ficheros cambiados (siempre que no colisionen con los cambios).
      \end{itemize}
    \end{itemize}
  \end{frame}

  \section{Para terminar.}

  \begin{frame}{Dudas}
    \dots
  \end{frame}
  
  \begin{frame}
    \frametitle{\begin{center}Fin\end{center}}
    \begin{center}
      \includegraphics[height=3cm]{kaleidos}
    \end{center}
  \end{frame}

\end{document}
