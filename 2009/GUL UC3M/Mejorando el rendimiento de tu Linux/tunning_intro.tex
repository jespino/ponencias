\section{Introducción}
\subsection{¿Qué es?}
\begin{frame}{¿Qué es?}
	\begin{itemize}
		\item Adaptar el equipo a nuestras necesidades.
		\item Equilibrar la capacidad y la demanda.
		\item Dimensionar correctamente nuestro hardware.
	\end{itemize}
\end{frame}

\subsection{¿Por qué?}
\begin{frame}{¿Por qué?}
	\begin{itemize}
		\item La configuración genérica no cubre las necesidades.
		\item El rendimiento es una pieza clave del sistema.
		\item No todos los recursos son igual de importantes.
	\end{itemize}
\end{frame}

\subsection{¿Cuándo?}
\begin{frame}{¿Cuándo?}
	\begin{itemize}
		\item Desde la planificación del proyecto.
		\item En cada una de las etapas del proyecto.
		\item Después de un backup!!!
	\end{itemize}
\end{frame}

\subsection{¿Dónde?}
\begin{frame}{¿Dónde?}
	\begin{itemize}
		\item En el hardware.
		\item En el software.
		\item En el entorno.
	\end{itemize}
\end{frame}

\subsection{¿Cómo?}
\begin{frame}{¿Cómo?}
	\begin{itemize}
		\item Fuertes conocimientos sobre el sistema.
		\item Monitorización.
		\item Benchmarks.
		\item Prueba y Error.
		\item Mas arte que ciencia.
		\item Cambiando:
		\begin{itemize}
			\item Parámetros del kernel.
			\item Parámetros del sistema de ficheros.
			\item Parámetros de los drivers.
			\item Ampliando hardware.
			\item Mejorando el entorno.
			\item Modificando el software (kernel incluido).
		\end{itemize}
	\end{itemize}
\end{frame}
