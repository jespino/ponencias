\section{Tunning}
\subsection{Herramientas}

\begin{frame}{Monitorizacion}
	\begin{itemize}
		\item top
		\item vmstat
		\item uptime
		\item ps
		\item free
		\item iostat
		\item mpstat
		\item numastat
		\item pmap
		\item netstat
		\item wireshark
		\item strace
		\item /proc
	\end{itemize}
\end{frame}

\begin{frame}{Benchmarks}
	\begin{itemize}
		\item lmbench
		\item iozone
		\item netperf
	\end{itemize}
\end{frame}

\subsection{En la CPU}
\begin{frame}{En la CPU}
	\begin{itemize}
		\item Herramientas: uptime, top, vmstat, lmbench\dots
		\item Determinar prioridades: Baja latencia, alta cantidad de proceso.
		\item Determinar las prioridades de los procesos.
		\item Buscar una disminución del context-switch por interrupción.
		\item Asignar interrupciones a una CPU en concreto (por ejemplo las de una tarjeta red).
		\item Podemos ampliar CPU.
	\end{itemize}
\end{frame}

\subsection{En la memoria}
\begin{frame}{En la memoria}
	\begin{itemize}
		\item Herramientas: top, vmstat, free, pmap, iostat, lmbench\dots
		\item Determinar prioridades: Aprovechamiento de la memoria, menor carga de proceso.
		\item Si hay mucho uso de swap, normalmente indica falta de memoria.
		\item Si hay muchas peticiones de paginas nos podemos plantear ampliar el tamaño de las mismas.
		\item Si tenemos poca memoria y muchos procesos muy pequeños, podemos disminuir el tamaño de pagina. 
		\item Podemos analizar procesos concretos para ver en que consumen la memoria.
		\item Podemos asegurarnos de usar programas con bibliotecas dinámicas.
		\item Podemos ampliar memoria.
	\end{itemize}
\end{frame}

\subsection{En el disco}
\begin{frame}{En la disco}
	\begin{itemize}
		\item Herramientas: vmstat, iostat, iozone, lmbench\dots
		\item Determinar prioridades: Aprovechamiento, menor carga de proceso, mas ancho de banda.
		\item Deberíamos determinar el sistema de fichero que nos interesa.
		\item Si usamos solo ficheros muy grandes, podríamos ampliar el tamaño de bloque del sistema de ficheros.
		\item Si usamos muchos ficheros pequeños, podríamos disminuir el tamaño de bloque del sistema de ficheros.
		\item Escoger el IO Scheduler que nos pueda interesar.
		\item Añadir mas RAM. Esto amplia la memoria de cache para disco.
		\item Establecer parámetros del sistema de ficheros (por ejemplo, noatime).
		\item Establecer las políticas de journaling correctas.
		\item Uso de sistemas RAID o volúmenes lógicos.
	\end{itemize}
\end{frame}

\subsection{En la red}
\begin{frame}{En la red}
	\begin{itemize}
		\item Herramientas: netperf, lmbench, netstat \dots
		\item Determinar prioridades: Ancho de banda, baja latencia.
		\item Ampliar el tamaño del frame (jumbo frames, no soportados por todo el hardware!).
		\item Asignar interrupciones a un procesador en concreto.
		\item Usar mejores tarjetas de red (para mas ancho de banda o para mas baja latencia).
		\item Hacer bonding o balanceo de carga.
		\item El firewall ralentiza las comunicaciones.
	\end{itemize}
\end{frame}
