\section{Introducción}
\subsection{¿Que es?}
\begin{frame}
	\begin{itemize}
		\item Núcleo del sistema operativo GNU/Linux.
		\item GPL.
		\item Estable.
		\item Maduro.
		\item Moderno.
		\item Eficiente.
		\item Versátil.
		\item Configurable.
		\item y mucho mas!!
	\end{itemize}
\end{frame}

\subsection{Un poco de historia}
\begin{frame}
	\begin{itemize}
		\item En Abril de 1991 Linus Torvals empieza el desarrollo de Linux.
		\item En Septiembre del 1991 lanza la versión 0.01
		\item En Diciembre del 1991 libera la primera versión ''self hosted''.
		\item En Marzo del 1994 aparece la versión 1.0.0.
		\item En 1996 aparece Tux como mascota de Linux y la versión 2.0 del mismo.
		\item En 1999 lanza la versión 2.2
		\item En 2001 lanza la versión 2.4
		\item En 2003 lanza la versión 2.6 (la actual).
	\end{itemize}
\end{frame}

\subsection{Numeración de versiones}
\begin{frame}
	\begin{block}{Hasta la versión 2.5}
		\begin{itemize}
			\item El primer numero indicaba la versión de la rama del kernel.
			\item El segundo indicaba si era estable o no (par estable, impar inestable).
			\item El tercero era la minor-release de esa rama.
		\end{itemize}
	\end{block}
	\begin{block}{versión 2.6.0 y posteriores}
		\begin{itemize}
			\item El primer y segundo numero se han estabilizado (no se esperan cambios a corto plazo).
			\item El tercer numero indica la versión del kernel.
			\item Si esta versión esta seguida por -rcX significa que es una versión de desarrollo.
			\item Si esta versión esta seguida por un .X significa que es estable y que esta en el bugfix X.
		\end{itemize}
	\end{block}
\end{frame}

\subsection{Git}
\begin{frame}{Git}
	\begin{itemize}
		\item Sistema de control de versiones del kernel de Linux.
		\item Distribuido.
		\item Desarrollado directamente por Linus Torvals.
		\item Apareció como una alternativa a BitKeeper.
		\item Existen muchas ramas desarrolladas de manera distribuida.
	\end{itemize}
\end{frame}

\subsection{Visión General}
\begin{frame}{A vista de pájaro}
	\begin{center}
		\includegraphics[height=6.5cm]{imgs/vista1}
	\end{center}
\end{frame}

\begin{frame}{Capas}
	\begin{center}
		\includegraphics[height=6.5cm]{imgs/vista2}
	\end{center}
\end{frame}

\begin{frame}{Subsistemas}
	\begin{center}
		\includegraphics[height=6.5cm]{imgs/vista3}
	\end{center}
\end{frame}
