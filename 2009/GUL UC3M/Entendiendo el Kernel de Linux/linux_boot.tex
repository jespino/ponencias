\section{Inicio del sistema}
\subsection{Inicio del sistema}
\begin{frame}{Inicio del sistema}
	\begin{itemize}
		\item Es el proceso que permite poner en funcionamiento al sistema operativo.
		\item Comienza por la BIOS, que realiza una comprobación del hardware y pasa el control a un programa que típicamente se encuentra en el sector de inicio del disco duro.
		\item El cargador de arranque se encarga de cargar el núcleo en memoria y pasarle el control junto con una serie de parámetros (se llama a la función \emph{setup( )} que se encuentra desplazada 0x200 bytes del inicio del núcleo).
	\end{itemize}
\end{frame}

\subsection{Proceso de inicio}
\begin{frame}{Proceso de inicio}
	El proceso de arranque realiza las siguientes tareas:
	\begin{itemize}
		\item Se invoca a la BIOS para saber la cantidad de RAM del sistema o, en sistemas ACPI, obtener un mapa físico de la RAM.
		\item Ajusta la la velocidad del repetición del teclado.
		\item Inicializa el adaptador de vídeo.
		\item Reinicia el controlador de disco y obtiene los parámetros de los discos.
		\item Comprueba que haya un bus IBM Micro Channel.
		\item Comprueba si hay un ratón PS/2.
		\item Comprueba si la BIOS soporta APM (\emph{Advanced Power Management}).
		\item En caso de que la BIOS soporte EDD (\emph{Enhanced Disk Drive Services}), se solicita crear una tabla en memoria con la descripción de los discos duros.
	\end{itemize}
\end{frame}
\begin{frame}{Proceso de inicio}
	\begin{itemize}
 		\item Según si el núcleo era sin comprimir o comprimido, procede a continuar o desplaza el núcleo en memoria para colocarlo donde le corresponde.
		\item Se activa el pin A20 en el controlador de teclado 8042.
		\item Se crea una tabla de descriptores de interrupciones (IDT) provisional y una tabla global de descripción (GDT) provisional.
		\item Reinicia la unidad de coma flotante (FPU).
		\item Se reprograma el controlador programable de interrupciones (PIC) para enmascarar todas las interrupciones menos la de IRQ2.
		\item Se pone la CPU en modo protegido.
		\item Se llama a la función \emph{startup\_32( )}.
	\end{itemize}
\end{frame}

\subsection{startup\_32}
\begin{frame}{La función startup\_32 (I)}
	\begin{itemize}
		\item Inicializa los registros de segmentación y una pila provisional.
		\item Pone a 0 los bits del registro \emph{eflags}.
		\item Inicializa con ceros el área de datos del núcleo que se encuentran en \emph{\_edata} y \emph{\_end}.
		\item Llama a \emph{decompress\_kernel( )}.
		\item Una vez descomprimido el núcleo se llama a un nuevo método \emph{startup\_32~(~)}.
	\end{itemize}
\end{frame}
\begin{frame}{La función startup\_32 (y II)}
	El nuevo \emph{startup\_32( )} hace las siguiente operaciones:
	\begin{itemize}
		\item Inicializa los registros de segmentación con los valores finales.
		\item Rellena el segmento \emph{bss} del núcleo con ceros.
		\item Inicializa la tabla de páginas provisional del núcleo.
		\item Almacena la dirección del \emph{Page Global Directory} en el registro \emph{cr3}.
		\item Asigna la pila de modo núcleo al proceso 0.
		\item Vuelve a poner a 0 el registro \emph{eflags}.
		\item Llama a \emph{setup\_idt( )}.
		\item Almacena los parámetros obtenidos de la BIOS en el primer \emph{frame} de memoria.
		\item Identifica el modelo del procesador.
		\item Carga los registros de la tabla GDT e IDT.
		\item salta a la función \emph{start\_kernel( )}. 
	\end{itemize}
\end{frame}

\subsection{start\_kernel}
\begin{frame}{La función start\_kernel}
	Llama a las siguientes funciones:
	\begin{itemize}
		\item \emph{sched\_init( )}.
		\item \emph{build\_all\_zonelists( )}.
		\item \emph{page\_alloc\_init( )} y \emph{mem\_init( )}.
		\item \emph{trap\_init( )}.
		\item \emph{softirq\_init( )}.
		\item \emph{time\_init( )}.
		\item \emph{kmem\_cache\_init( )}.
		\item \emph{calibrate\_delay( )}.
		\item \emph{kernel\_thread( )}.
	\end{itemize}
\end{frame}

