\documentclass{beamer}
\usepackage[spanish]{babel}
\usepackage[latin1]{inputenc}

%\usepackage{beamerouterthemeinfolines}
%\usepackage{beamerthemeBoadilla}      % Sin {,sub}apartados, blanco
%\usepackage{beamerthemeRochester}     % Sin {,sub}apartados, simple, cuadrado
%\usepackage{beamerthemeboxes}         % Sin {,sub}apartados, muy blanco
%\usepackage{beamerthemePittsburgh}    % Sin {,sub}apartados, muy blanco, redondo
%\usepackage{beamerthemeMadrid}        % Sin {,sub}apartados
%\usepackage{beamerthemeSingapore}     % Sin subapartados, blanco, puntitos
\usepackage{beamerthemeCopenhagen}    % Violeta/negro, cargado, con todo
%\usepackage{beamerthemeLuebeck}       % Como Copenhagen, cuadrado
%\usepackage{beamerthemeWarsaw}        % Violeta/negro, gradiente, cargado, con todo

% Image declarations
% \pgfdeclaremask{imagemask}{imagemaskfile}
% \pgfdeclareimage[interpolate=true,mask=imagemask,%
                 % width=2cm,height=1.3cm]{imagename}{imagename}
\pgfdeclareimage[interpolate=true,%
                 width=2cm]{gul}{gul}

\title{OpenVPN: Redes Privadas Virtuales sencillas}
\author{Jes�s Espino Garc�a\\ \texttt{jespinog@gmail.com}}
\institute{Jornadas de Oto�o 2009\\\pgfuseimage{gul}}

\date{\today}

\begin{document}

\frame{\titlepage}

\section<presentation>*{�ndice}

\frame
{
  \frametitle{�ndice}
  \tableofcontents[part=1]
}

\AtBeginSubsection[]
{
  \frame<handout:0>
  {
    \frametitle{�ndice}
    \tableofcontents[current,currentsubsection]
  }
}


\part<presentation>{Contents}

\section{Introducci�n}

%%%%%%%%%%%%%%%%%%%%%%%%%%%%%%%%%%%%%%%%%%%%%%%%%%%%%%%%%%%%%%%%%% VPN %
\frame
{
  \frametitle{�Qu� es una VPN?}
  \begin{itemize}
  \item Red Privada Virtual.
  \item Redes privadas sobre redes publicas.
  \item Trasferencia de datos cifrada.
  \end{itemize}
}

%%%%%%%%%%%%%%%%%%%%%%%%%%%%%%%%%%%%%%%%%%%%%%%%%%%%%%%%%%%%% OpenVPN %
\frame
{
  \frametitle{�Que es OpenVPN?}
  \begin{itemize}
  \item Software para construir VPNs.
  \item GPL.
  \item Multi Plataforma.
  \item Cifrado sobre OpenSSL.
  \end{itemize}
}

%%%%%%%%%%%%%%%%%%%%%%%%%%%%%%%%%%%%%%%%%%%%%%%%%%%%%%%%%%%%% Porque OpenVPN %
\frame
{
  \frametitle{�Por qu� OpenVPN?}
  \begin{itemize}
  \item Libre.
  \item Portable.
  \item SSL.
  \item F�cil de configurar.
  \item Escalable.
  \item Es un programa en espacio de usuario.
  \item Puede conectar a trav�s de proxy HTTP.
  \end{itemize}
}

\section{Usar OpenVPN}

\subsection{Modos de funcionamiento}

%%%%%%%%%%%%%%%%%%%%%%%%%%%%%%%%%%%%%%%%%%%%%%%%%%%%%%%%%%%%% Templates: �Que son? %
\frame
{
  \frametitle{Routing}
  \begin{itemize}
  \item Eficiente y escalable.
  \item Usa modo "Tun" del driver tun/tap.
  \item Modifica reglas de firewall y rutado.
  \item Funciona con IPv4 y en algunos casos con IPv6.
  \item No propaga el trafico broadcast.
  \end{itemize}
}

%%%%%%%%%%%%%%%%%%%%%%%%%%%%%%%%%%%%%%%%%%%%%%%%%%%%%%%%%%%%% Templates: �Que son? %
\frame
{
  \frametitle{Bridge}
  \begin{itemize}
  \item Bridge Ethernet.
  \item Usa modo "Tap" del driver tun/tap.
  \item Usa las bridge tools.
  \item Une dos redes.
  \item Soporta otros protocolos a parte de IPv4 (IPX, AppleTalk...).
  \item Menos eficiente y escalable que el routing.
  \end{itemize}
}

\subsection{Modos de cifrado}

%%%%%%%%%%%%%%%%%%%%%%%%%%%%%%%%%%%%%%%%%%%%%%%%%%%%%%%%%%%%% Templates: �Que son? %
\frame
{
  \frametitle{Clave compartida}
  \begin{itemize}
  \item Usar una clave que comparten cliente y servidor.
  \item Poco seguro.
  \item Poco escalable.
  \end{itemize}
}

%%%%%%%%%%%%%%%%%%%%%%%%%%%%%%%%%%%%%%%%%%%%%%%%%%%%%%%%%%%%% Templates: �Que son? %
\frame
{
  \frametitle{Certificados X509}
  \begin{itemize}
  \item Usar Certificados para autenticar ambos extremos.
  \item Utiliza una Autoridad certificador para dar validez a los certificados.
  \item Las altas se hacen mediante generaci�n y firma de certificados por parte de la CA.
  \item Las bajas se hacen como revocaciones de un certificado.
  \item Muy seguro.
  \item Muy escalable.
  \end{itemize}
}

\section{Ejemplos}
\subsection{Clave compartida con Routing}

%%%%%%%%%%%%%%%%%%%%%%%%%%%%%%%%%%%%%%%%%%%%% Templates: Ficheros y directorios %
\begin{frame}[fragile]
  \frametitle{Clave compartida con Routing}
  Configuraci�n con clave est�tica y compartida.
  \begin{itemize}
  \item Creamos una clave: \verb+openvpn --genkey --secret static.key+
  \item Copiamos esta clave en los dos equipos a conectar.
  \item Configuramos el servidor para que acepte esta clave.
  \item Configuramos el cliente para que conecte al servidor y use esta clave.
  \end{itemize}
\end{frame}


%%%%%%%%%%%%%%%%%%%%%%%%%%%%%%%%%%%%%%%%%%%%%%% Templates: Principales funciones %
\begin{frame}[fragile]
  \frametitle{Clave compartida con Routing}
  \begin{itemize}
  \item Cliente:
  \begin{verbatim}
dev tun0
ifconfig 10.8.0.1 10.8.0.2
secret static.key
  \end{verbatim}
  \item Servidor:
  \begin{verbatim}
remote myserver.com
dev tun0
ifconfig 10.8.0.2 10.8.0.1
secret static.key
  \end{verbatim}
  \end{itemize}
\end{frame}

\subsection{X509 con Bridging}

%%%%%%%%%%%%%%%%%%%%%%%%%%%%%%%%%%%%%%%%%%%%% Templates: Ficheros y directorios %
\begin{frame}[fragile]
  \frametitle{X509 con Bridging}
  Configuraci�n con certificados X509.
  \begin{itemize}
  \item Creamos una autoridad certificadora.
  \item Creamos un certificado y su clave para el servidor.
  \item Creamos un certificado y su clave para el cliente.
  \item Firmamos los certificados del servidor y cliente con la autoridad certificadora.
  \item Configuramos el servidor para usar esa CA, su certificado y su clave.
  \item Configuramos el cliente para usar esa CA, su certificado y su clave.
  \end{itemize}
\end{frame}


%%%%%%%%%%%%%%%%%%%%%%%%%%%%%%%%%%%%%%%%%%%%%%% Templates: Principales funciones %
\begin{frame}[fragile]
  \frametitle{X509 con Bridging}
  \begin{itemize}
  \item Cliente:
  \begin{verbatim}
client
dev tap1
proto tcp
port 1196
ca keys/ca.crt
cert keys/client.crt
key keys/client.key
dh keys/dh1024.pem
remote 127.0.0.1 1195
  \end{verbatim}
  \end{itemize}
\end{frame}

\begin{frame}[fragile]
  \frametitle{X509 con Bridging}
  \begin{itemize}
  \item Servidor:
  \begin{verbatim}
dev tap0
proto tcp
lport 1195
ca keys/ca.crt
cert keys/server.crt
key keys/server.key
dh keys/dh1024.pem
server-bridge 192.168.0.2 255.255.255.0 192.168.0.128 192.168.0.254
  \end{verbatim}
  \end{itemize}
\end{frame}

\section{Para terminar}

\subsection{Referencias}
%%%%%%%%%%%%%%%%%%%%%%%%%%%%%%%%%%%%%%%%%%%%%%%%%%%%%%%%%%%%%%%% Referencias %
\begin{frame}
  \frametitle{Referencias}
  \begin{itemize}
  \item www.openvpn.net: Pagina oficial de OpenVPN.
  \item gul@gul.uc3m.es: Lista de correo del GUL.
  \end{itemize}
\end{frame}

\subsection{Dudas}

%%%%%%%%%%%%%%%%%%%%%%%%%%%%%%%%%%%%%%%%%%%%%%%%%%%%%%%%%%%%%%%% Dudas %
\begin{frame}
  \frametitle{Dudas}
  \begin{center}
  \ldots
  \end{center}
\end{frame}

\end{document}
