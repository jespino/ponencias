\section{Esquemas e indices}

\subsection{Introducción}
\begin{frame}{Introducción}
  \begin{itemize}
    \item Nuestros esquemas e indices dependen de la funcionalidad.
    \item Importa tanto que vamos a almacenar como el como lo vamos a consultar.
    \item Todo el diseño de la base de datos debe contemplar los casos de uso.
    \item Es muy diferente una base de datos orientada a lectura, que una orientada a escritura.
    \item Es importante tener en cuenta las necesidades de latencias.
    \item Todo deriva en un compromiso entre el rendimiento en diferentes situaciones.
  \end{itemize}
\end{frame}

\subsection{Normalización}
\begin{frame}{Normalización}
  \begin{itemize}
    \item Reestructuración de nuestras tablas.
    \item Busca eliminar redundacia.
    \item Se aplican una serie de "formas normales".
    \item Normalmente es una buena politica.
  \end{itemize}
\end{frame}

\begin{frame}{Desnormalización}
  \begin{itemize}
    \item A veces la normalizacion es ineficiente.
    \item La redundancia puede producir incrementos de rendimiento significativos.
    \item Bases de datos con mucha lectura y poca escritura.
    \item Bases de datos con necesidades de latencias muy bajas.
  \end{itemize}
\end{frame}

\subsection{Filas de tamaño fijo}
\begin{frame}{Filas de tamaño fijo}
  \begin{itemize}
    \item Las filas almacenadas en nuestra base de datos tienen un tamaño.
    \item Este puede ser fijo o variable.
    \item Las filas de tamaño fijo son mas rapidas.
    \item Una fila se considera de tamaño fijo, si todos sus campos lo son.
    \item A veces conviene desaprovechar cierta cantidad de espacio a cambio de rendimiento.
    \item Dependiendo del tipo de consulta puede resultar mas rapido si ahorramos espacio.
    \item 
    \item 
  \end{itemize}
\end{frame}

\subsection{Indices}
\begin{frame}{Indices}
  \begin{itemize}
    \item Estructuras auxiliares para busquedas.
    \item Aceleran las consultas (cuando tienen datos suficientes).
    \item Pueden resolver la consulta entera (si los datos necesarios estan contenidos).
    \item Hacen referencia a uno o mas campos.
    \item Los indices de varios campos tienen un orden concreto.
    \item (A,B) != (B,A)
    \item Cada indice incrementa el espacio consumido y decrementa la velocidad de escritura.
  \end{itemize}
\end{frame}

\begin{frame}{Arboles B}
  \begin{itemize}
    \item Este es tipo de indice mas habitual.
    \item El formato interno del arbol depende del S.E.
    \item Este tipo de indice permite las siguientes consultas:
    \begin{itemize}
      \item El valor completo del indice.
      \item Valores en la parte izquierda del indice.
      \item Rangos de valores.
      \item Una parte exacta (a la izquierda), y el resto como un rango.
      \item Consultas de solo el indice.
    \end{itemize}
  \end{itemize}
\end{frame}

\begin{frame}{Tablas hash}
  \begin{itemize}
    \item Para cada columna, se calcula un hash y se asocia al indice.
    \item Solo permite busquedas exactas.
    \item Las busquedas son muy rapidas.
    \item Es el tipo por defecto del S.E. Memory.
    \item Este modo no esta disponible en MyISAM o InnoDB, pero se puede "emular".
  \end{itemize}
\end{frame}

\begin{frame}{Spatial indexes}
  \begin{itemize}
    \item Indices especiales para GIS.
  \end{itemize}
\end{frame}

\begin{frame}{Full text}
  \begin{itemize}
    \item Indices para busquedas sobre el contenido.
    \item Solo disponibles en MyISAM (por ahora).
  \end{itemize}
\end{frame}

\begin{frame}{Clustered indexes}
  \begin{itemize}
    \item No es otro tipo de index, es un concepto.
    \item Consiste en incluir los datos de la fila, dentro del indice de la clave primaria.
    \item Esto permite que la busqueda de la clave primaria de como resultado la fila, sin necesidad de ningun salto extra.
    \item InnoDB implementa este tipo de arbol B.
  \end{itemize}
\end{frame}

\begin{frame}{Coverage indexes}
  \begin{itemize}
    \item Extraer los datos directamente del indice.
    \item Solo si todos los datos estan contenidos en el indice.
    \item Supone un incremento significativo del rendimiento.
  \end{itemize}
\end{frame}
