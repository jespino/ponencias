\section{XFS}
\subsection{Introducción}
\begin{frame}{Introducción}
  \begin{itemize}
    \item Creado por Silicon Graphics para IRIX.
    \item Liberado bajo licencia GPL en Mayo del 2000.
    \item Actualmente incluido en el kernel de Linux
    \item Opcion en muchas distribuciones.
  \end{itemize}
\end{frame}

\subsection{Caracteristicas}
\begin{frame}{Caracteristicas}
  \begin{itemize}
    \item Journaling (Solo para los metadatos)
    \item ACLs
    \item Permisos POSIX
    \item Tamaño maximo: 9EB
    \item Tamaño maximo de fichero: 9EB
    \item Maximo de caracteres de nombre de fichero: 255B
    \item Maximo numero de ficheros: NA
  \end{itemize}
\end{frame}

\subsection{Estructura}
\begin{frame}{Estructura}
  \begin{itemize}
    \item En XFS no existe el espacio para el sector de arranque.
    \item El espacio se divide completamente en “Grupos de Asignacion”
    \item Cada grupo de asignacion esta compuesto por:
    \begin{itemize}
      \item El superbloque (Si no es el primer grupo, solo es una copia).
      \item Un espacio para informacion de asignacion de bloques e inodos del grupo (almacenada en arboles B+).
      \item Y un bloque de datos donde se almacenan los inodos y bloques de datos.
    \end{itemize}
  \end{itemize}
\end{frame}

\begin{frame}{Estructura}
  \begin{center}
    \includegraphics[height=5.5cm]{imgs/xfs_struct.png}
  \end{center}
\end{frame}

\begin{frame}{Ficheros}
  \begin{itemize}
    \item Los inodos contienen un nucleo con la informacion basica.
    \item Para especificar los bloques de datos usa “extents”, es decir, direccion inicial y tamaño total.
    \item Los inodos pueden tener almacenar los datos del fichero de tres maneras:
    \begin{itemize}
      \item LOCAL: Los datos directamente en el inodo.
      \item EXTENDS: Un mapa de extends de bloques de datos.
    \end{itemize}
  \end{itemize}
\end{frame}

\begin{frame}{Ficheros}
  \begin{center}
    \includegraphics[height=6cm]{imgs/xfs_files.png}
  \end{center}
\end{frame}
