\section{Ext 2/3/4}
\subsection{Introducción}
\begin{frame}{Introducción}
  \begin{itemize}
    \item Surgio como sustituo para el sistema Minix en Linux.
    \item En abril del 92 fue liberado la version 1 usando ya la API VFS de Linux.
    \item Soluciono gran parte de los problemas que planteaba Minix (Principalmente tamaño maximo y numero de caracteres).
    \item En el 93 aparece su version 2 incorporando ideas de Berkeley Fast File System y pensado para escabilidad.
  \end{itemize}
\end{frame}

\subsection{Caracteristicas}
\begin{frame}{Caracteristicas ext2}
  \begin{itemize}
    \item Permisos POSIX
    \item ACLs
    \item Tamaño maximo: 4TB
    \item Tamaño maximo de fichero: 2GB
    \item Maximo de caracteres de nombre de fichero: 255B
    \item Maximo numero de ficheros: NA
  \end{itemize}
\end{frame}

\begin{frame}{Caracteristicas ext3}
  \begin{itemize}
    \item Journaling
    \item ACLs
    \item Tamaño maximo: 4TB
    \item Tamaño maximo de fichero: 2GB
    \item Maximo de caracteres de nombre de fichero: 255B
    \item Maximo numero de ficheros: NA
  \end{itemize}
\end{frame}

\begin{frame}{Caracteristicas ext4}
  \begin{itemize}
    \item Journaling
    \item ACLs
    \item extends
    \item Tamaño maximo: 4TB
    \item Tamaño maximo de fichero: 2GB
    \item Maximo de caracteres de nombre de fichero: 255B
    \item Maximo numero de ficheros: NA
  \end{itemize}
\end{frame}

\subsection{Estructura}
\begin{frame}{Estructura}
  \begin{itemize}
    \item Al principio de la particion nos encontramos el sector de arranque.
    \item El resto del espacio (practicamente todo) se divide en grupos de bloques.
    \item Cada grupo de bloques contiene:
    \begin{itemize}
      \item Si es el primero un superbloque, si no, puede o no contener un backup de el superbloque.
      \item Un descriptor de grupo, que contiene informacion sobre el grupo de bloques.
      \item Un mapa de bits de bloques, que contiene informacion sobre los bloques libres.
      \item Un mapa de bits de inodos, que contiene infomacion sobre los inodos libres.
      \item Bloque de datos, que contiene los datos de los ficheros.
    \end{itemize}
  \end{itemize}
\end{frame}

\begin{frame}{Estructura}
  \begin{center}
    \includegraphics[height=5.5cm]{imgs/ext_struct.png}
  \end{center}
\end{frame}

\begin{frame}{Ficheros}
  \begin{itemize}
    \item La informacion de los ficheros se almacena en los inodos. Un fichero puede ser un directorio, un fichero regular, un socket, etc...
    \item En el inodo no se almacenan datos, solo punteros a los bloques de datos.
    \item Los punteros a los bloques de datos pueden ser:
    \begin{itemize}
      \item Directos: Apuntan directamente a un bloque de datos
      \item Indirectos: Apuntan a un bloque de punteros a bloques de datos.
      \item Doblemente indirectos: Apuntan a un bloque de punteros a bloques de punteros a bloques de datos.
      \item Triplemente indirectos: Apuntan a un bloque de puntersoa a bloques de punteros a bloques de punteros a bloques de datos.
    \end{itemize}
    \item Los datos almacenados en los directorios son el nombre del fichero y el inodo que lo contiene.
  \end{itemize}
\end{frame}

\begin{frame}{Ficheros}
  \begin{center}
    \includegraphics[height=6cm]{imgs/ext_files.png}
  \end{center}
\end{frame}
