\documentclass[10pt]{beamer}

\usepackage[utf8]{inputenc}
\usepackage[spanish]{babel}
\usepackage{graphicx}

\mode<presentation>
\usetheme{Madrid}
%\usecolortheme[RGB={111,73,135}]{structure}
\usecolortheme[RGB={128,0,0}]{structure}
%\usecolortheme[RGB={0,96,0}]{structure}
%\usecolortheme[RGB={200,0,200}]{structure}
%\usecolortheme[RGB={0,128,0}]{structure}
%\usecolortheme[RGB={0,0,128}]{structure}
\usefonttheme{serif}
\useinnertheme{rectangles}
\useoutertheme{split}
\setbeamercovered{transparent}


% Definiciones para usar luego
\newtheorem{ejemplo}{Ejemplo}
\newtheorem{definicion}{Definición}

\title{Fabric}
\author{Jesús Espino García}
\date{8 de Marzo de 2011}
\subject{Fabric}

\institute[GUL UC3M]{Python-ESP-Centro}

\setcounter{tocdepth}{2}

\AtBeginSubsection[]
{
  \begin{frame}<beamer>{Indice}
    \tableofcontents[sectionstyle=show/shaded,subsectionstyle=show/shaded/hide]
  \end{frame}
}

\begin{document}

  \frame{\maketitle}


  \section*{Introducción}
  \begin{frame}{¿Qué es Fabric?}
    \begin{itemize}
      \item Herramienta de despliegue.
      \item Herramienta ejecución de tareas.
      \item Herramienta para ejecución remota.
    \end{itemize}
  \end{frame}
  
  \begin{frame}{¿Para qué sirve Fabric?}
    \begin{itemize}
      \item Despligue de aplicaciones en entornos complejos.
      \item Despligue de aplicaciones en entornos distribuidos.
      \item Ejecucion de tareas administrativas habituales.
      \item Ejecución de tareas administrativas en entornos distribuidos.
      \item Scripts de construccion.
      \item \dots{}
    \end{itemize}
  \end{frame}
  
  \begin{frame}{¿Por qué usar Fabric?}
    \begin{itemize}
      \item Nos evita errores.
      \item Nos evita inconsistencias.
      \item Nos facilita las tareas periodicas.

      \item Disminuye los tiempos despligue.
      \item Es Python!!
    \end{itemize}
  \end{frame}

  \begin{frame}{¿Como funciona Fabric?}
    \begin{itemize}
      \item Se define una serie de tareas en un fichero fabfile.py.
      \item En este fichero se definen una serie de funciones (tareas).
      \item Se ejecuta la aplicacion fab <tarea>
    \end{itemize}
  \end{frame}
  
  \section*{Entendiendo Fabric}
  \begin{frame}{Entorno}
    \begin{itemize}
      \item La ejecución de fabric depende del entorno definido.
      \item El entorno es un diccionario.
      \item Para cambiar el comportamiento modificamos el entorno.
      \item 
      \item 
      \item 
    \end{itemize}
  \end{frame}

  \begin{frame}{Modelo de ejecución}
    \begin{itemize}
      \item Construcción de una lista de tareas.
      \item Construcción de una lista de hosts.
      \item Se recorre la lista de tareas.
      \item Para cada hosts se ejecuta la tarea actual.
    \end{itemize}
  \end{frame}

  \begin{frame}{Lista de tareas}
    \begin{itemize}
      \item Fabric extra las tareas del fichero fabfile.py (Normalmente).
      \item Considera tarea todo objeto "Callable" que haya al importar el fabfile.py.
      \item No se consideran tareas las funciones/metodos del propio fabric.
      \item 
    \end{itemize}
  \end{frame}

  \begin{frame}{Lista de hosts}
    \begin{itemize}
      \item La lista de hosts se extra del entorno.
      \item Si la tarea tiene un decorador de hosts sobreescribe los hosts del entorno.
      \item Si se le pasa como parametro los hosts sobreescribe todo lo anterior.
      \item 
    \end{itemize}
  \end{frame}

  \begin{frame}{Tareas con parametros}
    \begin{itemize}
      \item Las tareas pueden recibir parametros.
      \item 
      \item 
      \item 
    \end{itemize}
  \end{frame}
  
  \subsection*{La API de Fabric}
  \begin{frame}{Core}
    \begin{itemize}
      \item 
      \item 
      \item 
      \item 
    \end{itemize}
  \end{frame}

  \begin{frame}{Contrib}
    \begin{itemize}
      \item 
      \item 
      \item 
      \item 
    \end{itemize}
  \end{frame}
  
  \subsection*{Ejemplos}
  \begin{frame}{}
    \begin{itemize}
      \item 
      \item 
      \item 
      \item 
    \end{itemize}
  \end{frame}

  \section*{Para terminar}

  \begin{frame}{Dudas}
    \begin{center}
      \dots
    \end{center}
  \end{frame}

\end{document}
