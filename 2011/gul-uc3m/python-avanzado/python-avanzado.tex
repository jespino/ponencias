\documentclass[10pt]{beamer}

\usepackage[utf8]{inputenc}
\usepackage[spanish]{babel}
\usepackage{graphicx}

\mode<presentation>
\usetheme{Madrid}
%\usecolortheme[RGB={111,73,135}]{structure}
\usecolortheme[RGB={128,0,0}]{structure}
%\usecolortheme[RGB={0,96,0}]{structure}
%\usecolortheme[RGB={200,0,200}]{structure}
%\usecolortheme[RGB={0,128,0}]{structure}
%\usecolortheme[RGB={0,0,128}]{structure}
\usefonttheme{serif}
\useinnertheme{rectangles}
\useoutertheme{split}

\setbeamercovered{transparent}

\title{Python (un poco más) avanzado}
\author{Jesús Espino García\\jespinog@gmail.com\\@jespinog}
\date{7 de Noviembre de 2011}
\subject{Python (un poco más) avanzado}

\institute[Gul UC3M]{Gul UC3M Nov 2011\\
    \includegraphics[height=1.5cm]{kaleidos.png}
}

\setcounter{tocdepth}{2}

\AtBeginSection[]
{
  \begin{frame}<beamer>{Indice}
    \tableofcontents[sectionstyle=show/shaded,subsectionstyle=show/shaded/hide]
  \end{frame}
}

\begin{document}

  \frame{\maketitle}

  \section{Las bibliotecas de Python}
  \begin{frame}{\dots{}}
    \begin{itemize}
      \item http://docs.python.org/library/index.html
      \item http://pypi.python.org
    \end{itemize}
  \end{frame}
  
  \section{Usando python}
  \begin{frame}[containsverbatim]
    \frametitle{Comprehensions}
    \begin{verbatim}
>>> [ x*10 for x in range(1,10) if x%2==0 ]
[20, 40, 60, 80]

>>> [ [y for y in range(1,x)] for x in range(1,10) ]
[[],
 [1],
 [1, 2],
 [1, 2, 3],
 [1, 2, 3, 4],
 [1, 2, 3, 4, 5],
 [1, 2, 3, 4, 5, 6],
 [1, 2, 3, 4, 5, 6, 7],
 [1, 2, 3, 4, 5, 6, 7, 8]]
    \end{verbatim}
  \end{frame}

  \begin{frame}[containsverbatim]
    \frametitle{Comprehensions}
    \begin{verbatim}
# A partir de 2.7
>>> { x:x*10 for x in range(1,10) } 
{1: 10, 2: 20, 3: 30, 4: 40, 5: 50, 6: 60, 7: 70, 8: 80, 9: 90}
    \end{verbatim}
  \end{frame}

  \begin{frame}[containsverbatim]
    \frametitle{Funciones lambda}
    \begin{verbatim}
def pow(x):
    return x*x
    \end{verbatim}
    Es equivalente a:
    \begin{verbatim}
pow = lambda x: x*x
    \end{verbatim}
  \end{frame}

  \begin{frame}[containsverbatim]
    \frametitle{Closures}
Funciones que se aplican a una lista de elementos:
    \begin{verbatim}
>>> filter(lambda x: x>5, range(1,10))
[6, 7, 8, 9]

>>> map(lambda x: x*x, range(1,10))
[1, 4, 9, 16, 25, 36, 49, 64, 81]

>>> reduce(lambda x,y: x+y, range(1,10))
45

>>> sorted([7,4,5,9,1,3,2,6,8], lambda x,y: cmp(y,x))
[1, 4, 9, 16, 25, 36, 49, 64, 81]
    \end{verbatim}
  \end{frame}

  \begin{frame}[containsverbatim]
    \frametitle{Herencia multiple}
    \begin{verbatim}
# Estilo nuevo (de manera directa o indirecta
# hereda de object)
class ClaseDerivada(Base1, Base2, Base3):
    def metodo(self):
        super(ClaseDerivada, self).metodo()

# Estilo antiguo (no hereda de object)
class ClaseDerivada(Base1, Base2, Base3):
    def metodo(self):
        Base1.metodo()
    \end{verbatim}
  \end{frame}

  \begin{frame}[containsverbatim]
    \frametitle{Context managers}
    \begin{verbatim}
from contextlib import closing

with closing(file('fichero.txt','r')) as fichero:
    for line in fichero:
        print line
    \end{verbatim}
  \end{frame}

  \section{Extendiendo python}

  \begin{frame}[containsverbatim]
    \frametitle{Context managers}
    \begin{verbatim}
class MyManager(object):
    def __enter__(self):
        print "Entrando en el contexto"
    def __exit__(self, *args):
        print "Saliendo del contexto"

with MyManager():
    print "Dentro del contexto"
    \end{verbatim}
  \end{frame}

  \begin{frame}[containsverbatim]
    \frametitle{Generadores e iteradores}
    \begin{verbatim}
def generador():
    for x in range(1,10):
        yield x
class Iterador(object):
    def __iter__(self):
        for x in range(1,10):
            yield x
    \end{verbatim}
  \end{frame}

  \begin{frame}[containsverbatim]
    \frametitle{Generadores e iteradores}
    \begin{verbatim}
class Iterador2(object):
    def __init__(self):
        self.max = 10
        self.counter = 0

    def __iter__(self):
        return self

    def next(self):
        self.counter += 1
        if self.counter<self.max:
            return self.counter
        else:
            raise StopIteration
    \end{verbatim}
  \end{frame}

  \begin{frame}[containsverbatim]
    \frametitle{Metaclases}
    Sirven para definir nuestros propios tipos
    \begin{verbatim}
class Meta(type):
   def __getattribute__(*args):
      print "Metaclass getattribute invoked"
      return type.__getattribute__(*args)

class C(object):
    __metaclass__ = Meta
    def __len__(self):
        return 10
    def __getattribute__(*args):
        print "Class getattribute invoked"
        return object.__getattribute__(*args)
    \end{verbatim}
  \end{frame}

  \begin{frame}[containsverbatim]
    \frametitle{Metaclases}
    \begin{verbatim}
>>> c = C()
>>> c.__len__()                 # Explicit lookup via instance
Class getattribute invoked
10
>>> type(c).__len__(c)          # Explicit lookup via type
Metaclass getattribute invoked
10
>>> len(c)                      # Implicit lookup
10
    \end{verbatim}
  \end{frame}

  \section{El ecosistema python}

  \begin{frame}[containsverbatim]
    \frametitle{Gestionar software}
    \begin{verbatim}
$ pip search django
$ pip install Django
$ pip uninstall Django
$ pip install Django==1.3
    \end{verbatim}
  \end{frame}

  \begin{frame}[containsverbatim]
    \frametitle{Entornos virtuales}
    \begin{verbatim}
$ mkvirtualenv mi_directorio
$ cd mi_directorio
$ source bin/activate
$ pip install Django==1.3
$ cd ..
$ mkvirtualenv otro_directorio
$ cd otro_directorio
$ source bin/activate
$ pip install Django==1.2
$ deactivate
    \end{verbatim}
  \end{frame}

  \begin{frame}[containsverbatim]
    El sistema de empaquetado de python se basa en un fichero setup.py
    \frametitle{Empaquetar bibliotecas}
    \begin{verbatim}
#!/usr/bin/env python

from distutils.core import setup

setup(name='MiBiblioteca',
      version='1.0',
      description='Da charlas para que yo me pueda quedar tomando unas cervezas',
      author='Jesús Espino',
      author_email='jesus.espino@kaleidos.net',
      url='http://www.kaleidos.net',
      packages=['mibiblioteca'],
     )
    \end{verbatim}
http://docs.python.org/distutils/setupscript.html
  \end{frame}

  \begin{frame}[containsverbatim]
    \frametitle{Debugging (PDB)}
    \begin{verbatim}
import pdb; pdb.set_trace()
    \end{verbatim}
  \end{frame}

  \begin{frame}[containsverbatim]
    \frametitle{IPython y IPDB}
    Interpretes ricos de python y pdb.
  \end{frame}

  \begin{frame}[containsverbatim]
    \frametitle{Profiling}
    Desde el codigo:
    \begin{verbatim}
import cProfile
cProfile.run('foo()')
    \end{verbatim}
    Desde el interprete:
    \begin{verbatim}
$ python -m cProfile myscript.py
    \end{verbatim}
  \end{frame}

  \begin{frame}[containsverbatim]
    \frametitle{Cython}
    \begin{verbatim}
from libc.math cimport sin

cdef double f(double x):
    return sin(x*x)

    def hellocython(x):
        return "Hola cython %f" % (f(x))
    return sin(x*x)
    \end{verbatim}
  \end{frame}

  \section{Para terminar}
  
  \begin{frame}{Referencias}
    \begin{itemize}
      \item Proyecto python: http://www.python.org.
      \item Documentación de python: http://docs.python.org.
      \item Python-ES: python-es@python.org
    \end{itemize}
  \end{frame}

  \begin{frame}{Dudas}
    \begin{center}
      \dots
    \end{center}
  \end{frame}

\end{document}
