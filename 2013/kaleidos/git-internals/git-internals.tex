\documentclass[10pt]{beamer}

\usepackage[utf8]{inputenc}
\usepackage[spanish]{babel}
\usepackage{graphicx}

\mode<presentation>
\usetheme{Madrid}
%\usecolortheme[RGB={111,73,135}]{structure}
\usecolortheme[RGB={128,0,0}]{structure}
%\usecolortheme[RGB={0,96,0}]{structure}
%\usecolortheme[RGB={200,0,200}]{structure}
%\usecolortheme[RGB={0,128,0}]{structure}
%\usecolortheme[RGB={0,0,128}]{structure}
\usefonttheme{serif}
\useinnertheme{rectangles}
\useoutertheme{split}

\setbeamercovered{transparent}

\title{Git Internals}
\author{Jesús Espino García}
\date{7 de Abril de 2013}
\subject{Git Internals}

\institute[Kaleidos]{\includegraphics[height=1.5cm]{kaleidos.png}}

\setcounter{tocdepth}{2}

\AtBeginSubsection[]
{
  \begin{frame}<beamer>{Indice}
    \tableofcontents[sectionstyle=show/shaded,subsectionstyle=show/shaded/hide]
  \end{frame}
}

\begin{document}

  \frame{\maketitle}

  \section*{Introducción}

  \begin{frame}
    \frametitle{¿Por qué?}
    \begin{itemize}
      \item La interfaz de git es de bajo nivel.
      \item Conocer git bien da mucho poder.
      \item El poder está ahí aunque no lo conozcamos.
      \item Un gran poder conlleva una gran responsabilidad.
    \end{itemize}
  \end{frame}

  \begin{frame}
    \frametitle{Conceptos básicos}
    \begin{itemize}
        \item Procelain (Porcelana).
        \item Plumbing (Cañerias).
        \item Objetos
        \item Referencias
        \item Head
        \item Working copy
        \item Stage
        \item Stash
    \end{itemize}
  \end{frame}

  \begin{frame}
    \frametitle{Contenido de .git}
    \begin{itemize}
        \item Ficheros.
        \begin{itemize}
            \item HEAD
            \item index
            \item config
        \end{itemize}
        \item Directorios.
        \begin{itemize}
            \item objects
            \item refs
            \item hooks
            \item info
        \end{itemize}
    \end{itemize}
  \end{frame}

  \section*{Objetos}

  \begin{frame}
    \frametitle{Objetos}
    \begin{itemize}
        \item Bloque de datos almacenado en git
        \item Referenciado por el sha1 de su contenido
        \item Almacenados en el directorio .git/objects/ (o en packs).
        \item Hay 4 tipos de objetos en git (blob, tree, commit, tag).
    \end{itemize}
  \end{frame}

  \begin{frame}
    \frametitle{blobs}
    \begin{itemize}
        \item Será el nodo hoja de nuestros arboles.
        \item Será equivalente (normalmente) a nuestros ficheros.
    \end{itemize}
  \end{frame}

  \begin{frame}
    \frametitle{trees}
    \begin{itemize}
        \item Es un directorio de referencias a blob y otros trees.
        \item Almacena referencias (sha1 de objetos) y metadatos.
    \end{itemize}
  \end{frame}

  \begin{frame}
    \frametitle{diagrama de ejemplo de trees}
    % TODO
  \end{frame}

  \begin{frame}
    \frametitle{commits}
    \begin{itemize}
        \item Almacena una referencia a un tree.
        \item Almacena una referencia a su commit padre.
        \item Almacena metadatos del commit (autor, fecha, mensaje...)
    \end{itemize}
  \end{frame}

  \begin{frame}
    \frametitle{diagrama de ejemplo de commits}
    % TODO
  \end{frame}

  \begin{frame}
    \frametitle{Objects storage}
    \begin{itemize}
        \item Se añade una cabecera con el tipo de objeto y la longitud del mismo.
        \item Se concatena con los datos que se van a almacenar
        \item Se calcula su sha1 que se utilizara como nombre del objeto.
        \item Se comprime con zlib.
        \item Se almacena en .git/objects/XX/XXXXXXXXXXXXXXXXXXXXXXXXXXXXXXXXXXXXXX
    \end{itemize}
  \end{frame}

  \section*{Referencias}

  \begin{frame}
    \frametitle{branchs}
  \end{frame}

  \begin{frame}
    \frametitle{HEAD}
  \end{frame}

  \begin{frame}
    \frametitle{tags}
  \end{frame}

  \begin{frame}
    \frametitle{remotes}
  \end{frame}

  \begin{frame}
    \frametitle{Refspects}
  \end{frame}

  \begin{frame}
    \frametitle{Pushing refspects}
  \end{frame}

  \begin{frame}
    \frametitle{Borrando referencias}
  \end{frame}

  \section*{Packfiles}

  \section*{Protocolos de transferencia}

  \begin{frame}
    \frametitle{Dumb protocolo (HTTP)}
  \end{frame}

  \begin{frame}
    \frametitle{Smart protocolo (Git/SSH/File)}
  \end{frame}

  \section*{Mantenimiento y recuperación de datos}

  \begin{frame}
    \frametitle{Garbage collector}
  \end{frame}

  \begin{frame}
    \frametitle{Recuperar datos}
  \end{frame}

  \begin{frame}
    \frametitle{Eliminar objetos}
  \end{frame}

  \section*{Hablemos de algunos comandos}

  \begin{frame}
    \frametitle{init}
  \end{frame}

  \begin{frame}
    \frametitle{add}
  \end{frame}

  \begin{frame}
    \frametitle{commit}
  \end{frame}

  \begin{frame}
    \frametitle{push}
  \end{frame}

  \begin{frame}
    \frametitle{fetch}
  \end{frame}

  \begin{frame}
    \frametitle{reset}
  \end{frame}

  \begin{frame}
    \frametitle{rebase}
  \end{frame}

  \begin{frame}
    \frametitle{checkout}
  \end{frame}

  \section*{Para terminar}

  \begin{frame}
    \frametitle{Referencias}
    \begin{itemize}
      \item \small{http://git-scm.com/ - Web oficial de git.}
      \item \small{http://git-scm.com/book - ProGit (El libro de Git).}
      \item \small{http://github.com/ - Servicio de git por excelencia.}
      \item \small{http://bitbucket.org/ - Servicio de git de repositorios privados gratis.}
    \end{itemize}
  \end{frame}

  \begin{frame}
    \frametitle{Dudas}
    \dots
  \end{frame}

\end{document}
